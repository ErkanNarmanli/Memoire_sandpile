\section{Prémices d'algèbre}
	Introduction [TODO]
	
	Dans le reste de cette section, on fixera~$G$ \marginnote{$G,n$} un groupe abélien fini.
	
	Soient~$F: V \rightarrow G$ et~$C : V \rightarrow G^*$ deux homomorphismes, on définit les homomorphismes~$\phiCF : V \rightarrow G^* \oplus G$ et~$\phiFC : V \rightarrow G \oplus G^*$ en posant pour tout vecteur~$v$ de~$V$ :~$\phiCF(v) = (C(v),F(v))$ et~$\phiFC(v) = (F(v),C(v))$\marginnote{$\phiCF, \phiFC$}.
	
	Soient maintenant~$\gamma$ un réel tel que~$0<\gamma<1$ et~$F : V \rightarrow G$ un homomorphisme. On dit qu'un homomorphisme~$C : V \rightarrow G^*$ est $\gamma$-\emph{robuste}\marginnote{robuste} pour~$F$ lorsque que pour tout~$\sigma \subset [n]$ de cardinal~$\#\sigma < \gamma n$ on a~$\ker(\phiFC\restr{\Vsig}) \neq \ker(F\restr{\Vsig})$. Dans le cas contraire, on dit que~$C$ est~$\gamma$-\emph{faible}\marginnote{faible} pour~$F$. S'il n'y a pas d'ambiguïté on dira simplement \emph{robuste} ou \emph{faible}.
	
	\begin{lem}[Quantité d'homomorphismes faibles]\label{lem:3.1}
		Il existe une constante~$C_G$, qui ne dépend que de~$G$ et telle que : pour tout entier~$n$, pour tout réel~$\gamma$ strictement compris entre~$0$ et~$1$, et pour tout homomorphisme~$F : V \rightarrow G$ on ait :
		\[
			\#\{C\in\Hom(H,G^*), C\text{ faible pour}F\}
			\leq
			C_G \binom{n}{\entierSup{\gamma n} - 1} \# G^{\gamma n}.
		\] 
	\end{lem}
	\begin{dem}
		[TODO]
	\end{dem}
	On définit l'application d'évaluation~$e : G^* \times G \rightarrow R$\marginnote{$e$} par~$e((g^*,g)) = g^*(g)$. On définit enfin le morphisme~$t : (G\oplus G^*) \times (G^*\oplus G) \rightarrow R$ par
	\[
		t : ((g_1,\phi_1),(\phi_2,g_2) \mapsto (\phi_2(g_1) ,\phi_1(g_2))\text{\marginnote{$t$}}.
	\]
	De telle sorte que~$t(\phiCF(u), \phiFC(v)) = (e(C(u), F(v)), e(C(v), F(u)))$. On a alors le résultat suivant :
	\begin{lem}\label{lem:3.2}
		Soient~$F: V \rightarrow G$ et~$C : V \rightarrow G^*$ deux homomorphismes, et soit~$U$ un sous module de~$V$ tel que~$F(U)=G$. Si maintenant~$U'$ est un sous-module de~$V$ tel que l'application~$t$ soit nulle sur le sous-espace~$\phiCF(U)\times\phiFC(U')$ ; alors :
		
		Si~$p : G \oplus G^* \rightarrow G$ est la projection canonique, alors~$p\restr{\phiFC(U')}$ est injective. En particulier on a l'égalité :~$\ker(\phiFC\restr{U'}) = \ker(F\restr{U'})$.
	\end{lem}
	\begin{dem}
		[TODO]
	\end{dem}
	\begin{coro}\label{coro:3.3}
		Soient~$F: V \rightarrow G$ et~$C : V \rightarrow G^*$ deux homomorphismes, soit~$U$ un sous module de~$V$ tel que~$F(U)=G$. Dés lors, on a :
		\[
			\text{si } \#\{i\in[n], t(\phiCF(U), \phiFC(v_i)) \neq 0\} < \gamma n
			\text{ alors }C 
			\text{ est faible pour }F.
		\]
	\end{coro}
	\begin{dem}
		[TODO]
	\end{dem}