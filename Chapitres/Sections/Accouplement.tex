\section{Accouplements de dualité}
	Soient~$D_1$ un diviseur de~$G$ de degré nul. Puisque~$\Jac(G)$ est un groupe d'ordre fini, alors il existe~$m_1$ un entier positif non nul tel que la classe~$\overline{D_1}$ de~$D_1$ dans~$\Jac(G)$ vérifie~$m_1 \overline{D_1}=\overline{0}$ ; c'est à dire~$m_1 D_1 \in \Princ(G)$. Ainsi on prend~$m_1, m_2$ des entiers tels que :
	\begin{align*}
		m_1 D_1 &= \divi(f_1)\\
		m_2 D_2 &= \divi(f_2),
	\end{align*}
	où~$f_1, f_2$ sont deux applications méromorphes sur~$G$. Avec ces notations, on définit ensuite l'application~$\scalaire{\,\cdot\,}{\,\cdot\,} :\Div^0(G)\times\Div^0(G) \rightarrow \mathbb{Q}$ et donnée par :
	\[
		\scalaire{D_1}{D_2} = \frac{1}{m_2} \sum_{v\in V}D_1(v) f_2(v).	\text{\marginnote{$\scalaire{\,\cdot\,}{\,\cdot\,}$}}
	\]
	\begin{rem}
		L'application~$\scalaire{\,\cdot\,}{\,\cdot\,}$ est symétrique et bilinéaire. Le caractère bilinéaire est clair, pour démontrer le caractère symétrique il suffit de remplacer~$D_1(v)$ par~$\ord_v(f_1)/m_1$ dans la définition plus haut, puis de mettre ensemble les deux contributions des~$u,v$ voisins, ce qui nous donne une expression clairement symétrique en~$f_1,f_2$ :
		\[
			\scalaire{D_1}{D_2} = \frac{1}{m_1 \cdot m_2}
			\sum_{u \sim v}\Big(
				f_2(v)\left(f_1(u) - f_1(v)\right)
				+
				f_3(u)\left(f_1(v)-f_1(u)\right)
			\Big).
		\]
	\end{rem}
	\begin{rem}
		On remarque par ailleurs que si~$D_1$ (ou~$D_2$) est dans~$\Princ(G)$, alors~$m_1$ (ou~$m_2$) vaut~$1$ et donc~$\scalaire{D_1}{D_2}$ est dans~$\mathbb{Z}$. On peut donc passer au quotient, ce qui nous permet de définir :
		\[
			\delta_G : \Jac(G)\times\Jac(G) \rightarrow \quotient{\mathbb{Q}}{\mathbb{Z}}, \text{\marginnote{$\delta_G$}}
		\]
		comme quotient de~$\scalaire{\,\cdot\,}{\,\cdot\,}$. On appelle~$\delta_G$ l'\emph{accouplement canonique} associé à~$G$.
	\end{rem}
	\begin{defi}
		Une application~$\delta : \Gamma\times\Gamma \rightarrow \mathbb{Q}/\mathbb{Z}$ est dite \emph{parfaite} \marginnote{app. parfaite} lorsque pour tout~$x$ dans~$\Gamma$ on a : pour tout~$y$ dans~$\Gamma, \delta(x,y) = 0$ si et seulement si~$x = 0$.
		
		On dit que~$\delta$ est un \emph{accouplement} \marginnote{accouplement} lorsque~$\delta$ est symétrique, bilinéaire, et parfaite.
	\end{defi}
	\begin{prop}
		L'application~$\delta_G$ est un accouplement.
	\end{prop}
	\begin{dem}
		Il s'agit de montrer que~$\delta_G$ est une application parfaite, c'est à dire que pour~$x$ dans~$\Gamma$ on a~$\delta_G(x,\cdot) = 0$ si et seulement si~$x$ est nul.
		
		Pour le sens direct : si pour tout diviseur~$D_2$ on a que~$\scalaire{D_1}{D_2}$ est un entier, alors en particulier en prenant les diviseurs~$D_2 = (v_k) - (v_1)$ pour~$2\leq k \leq n$ on a pour tout entier~$k$ entre~$2$ et~$n$ que~$(f_1(v_k)-f_1(v_1))/m_1$ est un entier, que l'on notera~$f(k)$. Cela revient à dire que l'on peut écrire :
		\[
			\frac{f_1}{m_1} = \frac{f_1(v_1)}{m_1} + f \quad , f\in\mathcal{M}(G).
		\]
		Dès lors on peut passer au diviseurs dans l'expression ci-dessus, sachant que~$\divi(f+c) = \divi(f)$ pour toute constante~$c$, cela nous donne que~$D_1 = \frac{1}{m_1}\divi(f_1) = \divi(f)$ est principal, ce qui est bien ce que nous souhaitions montrer.
		
		La réciproque est triviale puisque~$\delta_G$ est bilinéaire.
	\end{dem}
	Aussi, plutôt que de travailler simplement avec~$\Jac(G)$, on préfère travailler avec le couple~$(\Jac(G),\delta_G)$ parce que la répartition est pus facile à étudier ainsi. Ceci nous amène donc à poser :
	\[
		A = \{(\Gamma,\delta), \Gamma\text{ abélien fini}, \delta\text{ accouplement}\}.\text{\marginnote{$A$}}
	\]
	Par la suite on dit que deux couples~$(\Gamma,\delta)$ et~$(\Gamma',\delta')$ sont équivalents lorsque :~$\Gamma$ et~$\Gamma'$ sont isomorphes et qu'il existe un isomorphisme~$f$ entre~$\Gamma'$ et~$\Gamma$ qui préserve l'accouplement, à savoir que pour tout~$x,y$ dans~$\Gamma'$, on a~$\delta(f(x),f(y)) = \delta'(x,y)$. Dans le cas où~$\delta = \delta'$, on notera~$\Aut(\Gamma,\delta)$\marginnote{$\Aut(\Gamma,\delta)$} l'ensembles des tels~$f$.
	
	On note~$\sim$ cette relation d'équivalence. On pose alors~$\mathscr{A} = \quotient{A}{\sim}$\marginnote{$\sim, \mathscr{A}$} l'ensemble quotient induit par~$\sim$. On pose également~$\mathscr{A}(m)$ \marginnote{$\mathscr{A}(m)$} l'ensemble des éléments~$(\Gamma,\delta)$ de~$\mathscr{A}$ tels que~$\Gamma$ soit de cardinal~$m$.