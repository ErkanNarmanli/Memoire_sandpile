\newpage
\section*{Conclusion}
On a donc finalement prouv\'e un r\'esultat concernant les matrices sym\'etriques dans~$\Zp$, ce qui est \'eloign\'e de notre but initial. Cependant, ce r\'esultat a tout de m\^eme des cons\'equences int\'eressantes.

Tout d'abord, la constante de normalisation (le num\'erateur du th\'eor\`eme) tend vers~$c_{p} = \prod_{i}{(1-p^{-i})}$ quand~$n$ tend vers~$+\infty$, ce qui nous donne par le th\'eor\`eme de convergence domin\'ee l'existence d'une mesure~$\check \eta$ sur~$\mathcal A_{p}$ telle que :
\[ \check \eta[(\Gamma, \delta)] \propto \frac1{|\Gamma|\cdot|\Aut(\Gamma,\delta)|} \]

C'est un bon d\'ebut dans la recherche de la distribution des~$p$-Sylows de la jacobienne ; et en effet, ce qui reste \`a d\'emontrer (et ce que M. Wood d\'emontre dans \cite{main}, sans consid\'erer l'accouplement canonique) est que la distribution du laplacien d'un graphe al\'eatoire dans~$\Zp$ est similaire \`a la mesure de Haar sur les matrices sym\'etriques.

Cet article est donc le point central pour d\'emontrer la convergence faible de~$\mu_{n}\circ\alpha_{p}^{-1}$ vers~$\check \eta_{p}$, d\'emontr\'ee uniquement pour l'instant pour les fonctions ne d\'ependant pas de l'accouplement $\delta$.
