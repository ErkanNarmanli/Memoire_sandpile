\section{Quelques lois de probabilité}
	On fixe dans le reste de cet exposé~$q$ un réel strictement compris entre~$0$ et~$1$. On notera~$\tilde{\mu}_n$\marginnote{$\tilde{\mu}_n$} la loi des graphes d'Erd\H{o}s-Rényi sur l'ensemble des graphes à~$n$ sommets. On a donc
	\[
		\tilde{\mu}_n(G) = q^{e(G)}(1-q)^{\binom{n}{2}-e(G)},
	\]
	où~$e(G)$ désigne le nombre d'arrêtes de~$G$. On pose alors~$J:G(n,q)\rightarrow \mathscr{A}$ l'application qui à tout graphe~$G$ associe le couple~$(\Jac(G),\delta_G)$. On pose enfin~$\mu_n = \tilde{\mu}_n \circ J^{-1}$\marginnote{$\mu_n$} la mesure image de~$\tilde{\mu}_n$ par~$J$.
	\begin{heur}
		On à l'heuristique suivant :~$\displaystyle\Pro((\Jac(G),\delta_G)=(\Gamma,\delta) \propto \frac{1}{\#\Gamma\#\Aut(\Gamma,\delta)}\cdot$
	\end{heur}
	Ceci nous amène alors définir la mesure~$\eta_n$ sur~$\mathscr{A}(1)\sqcup ...\sqcup \mathscr{A}(n)$ selon cet heuristique, on pose pour tout~$(\Gamma,\delta)$ dans~$\mathscr{A}$ tel que~$\#\Gamma \leq n$ :
	\[
		\eta_n(\Gamma,\delta) \propto\frac{1}{\#\Gamma\#\Aut(\Gamma,\delta)}\cdot
		\text{\marginnote{$\eta_n$}}
	\]
	Cela est possible puisque l'on est sur un ensemble fini. Le but est donc maintenant de comparer~$\eta_n$ et~$\mu_n$ asymptotiquement, au sens suivant. Soient~$\suite{\alpha},\suite{\beta}$ une suite de mesures sur un ensemble mesurable~$(E,\mathcal{E})$, on dit que les deux suites sont \emph{faiblement équivalentes}\marginnote{éq. faible} lorsque pour toute fonction~$F : E\rightarrow\mathbb{R}$ bornée et~$\mathcal{E}$-mesurable les deux propriétés suivantes sont vérifiées :
	\begin{enumerate}
		\item On a :~$\int F\,\mathrm{d}\alpha_n$ converge si et seulement si~$\int F\,\mathrm{d}\beta_n$ converge ;
		\item Si ces intégrales convergent, alors elles ont la même limite.
	\end{enumerate}
	 On a alors la conjecture suivante :
	\begin{conj}
		Les suites~$\mu_n$ et~$\eta_n$ sont faiblement équivalentes.
	\end{conj}
	Pour notre part, nous allons restreindre notre étude au cas où~$\Gamma$ est un~$p$-groupe et donc regarder le~$p$-Sylow de la Jacobienne. D'une part cela nous permet d'étudier des objets plus simples, et d'autre part cela nous fournit toujours beaucoup d'information puisque les~$p$-Sylow déterminent uniquement la Jacobienne. Cela nous amène donc à poser :
	\[
		\mathscr{A}_p = \bigsqcup_{n\in\mathbb{N}} \mathscr{A}(n)
			= \{(\Gamma,\delta), \exists n\in\mathbb{N},\#\Gamma = p^n\}.
			\text{\marginnote{$\mathscr{A}_p$}}
	\]
	Par la suite, si~$\Gamma$ est un groupe abélien fini, on notera~$\Gamma_p$ son~$p$-Sylow. On défini alors l'application ~$\alpha_p : \mathscr{A} \rightarrow\mathscr{A}_p$\marginnote{$\alpha_p$} qui associe à tout couple~$(\Gamma,\delta)$ de~$\mathscr{A}$ son image~$(\Gamma_p,\delta\restr{\Gamma_p})$. Cela nous amène alors à présenter une seconde conjecture :
	\begin{conj}
		La suite de mesures~$\mu_n\circ\alpha_p^{-1}$ converge faiblement vers une mesure~$\check{\eta}$, où~$\check{\eta}(\Gamma_p,\delta)$ est proportionnel à~$(\#\Gamma_p\#\Aut(\Gamma_p,\delta))^{-1}$.
	\end{conj}
	Dans son article \emph{The distribution of sandpile groups of random graphs} \cite{main},  Mélanie Matchett Wood montre un résultat similaire, en oubliant la notion d'accouplement. Elle montre le théorème suivant pour tout~$p$-groupe~$\Gamma-p$ :
	\begin{thm}
		On a~$\displaystyle \Pro((\Jac(G))_p\simeq \Gamma_p) \propto \frac{\#\{\text{accouplements }\Gamma_p\times \Gamma_p\rightarrow \Gamma_p\}}{\#\Gamma_p \cdot \#\Aut(\Gamma_p)} \cdot$
	\end{thm}
	Pour note part, nous allons nous intéresser à montrer un résultat de cette forme, non pas pour le cas de jacobienne (donc de conoyau de laplacien), mais dans le cas plus général des conoyaux de matrices symétriques. On travaillera sur des matrices à coefficients dans les \emph{nombres}~$p$-\emph{adiques}. Dans la sections suivantes nous allons voir quelques propriétés de tels objets.
	