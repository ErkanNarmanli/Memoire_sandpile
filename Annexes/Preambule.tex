\documentclass[a4paper, twoside, 11pt]{report}

% Classes possibles : article, amsart, report, letter, moderncv,...
%
% Options possibles : 10pt, 11pt, 12pt (taille de la fonte)
%                     oneside, twoside (recto simple, recto-verso)
%                     draft, final (stade de développement)
%                     titlepage, notitlepage (\maketitle prend une page
%                                             à part ou non)
%===========================================PACKAGES================================================
\usepackage[utf8]{inputenc}				%Codage d'entrée, ici l'Unicode UTF8
\usepackage[T1]{fontenc}      				%Codage de sortie, ici le type 1
%\usepackage{xelibertine}				%fonte utilisée, ici Libertine (max une fonte)
\usepackage{lmodern}					%fonte utilisée, ici Latin Modern (max une fonte)
\usepackage[francais]{babel} 				%Gère les langues, la langue par défaut est la dernière
							%|passé en argument
											
\usepackage[a4paper]{geometry}				%Permet de spécifier les dimensions du texte
\usepackage{babelbib}
%\usepackage{makeidx}\makeindex				%Permet de faire un index
%\usepackage[french]{minitoc}				%Pour faire des 'sous tables des matières'
%\usepackage{fancyhdr}						%Pour les en-tête et pieds de page
%\usepackage{titlesec}						%Deux paquets pour modifier l'affichage des parties
%\usepackage{titletoc}						%|
\usepackage[refpage,french,intoc]{nomencl}			%Pour ajouter un Index
	\makenomenclature					%|active l'index
	\renewcommand{\pagedeclaration}[1]{\dotfill \hyperpage{#1}}

\usepackage[colorlinks=true,urlcolor=magenta,linkcolor=black,citecolor=black]{hyperref}				
								%Permet de créer des références cliquables
\usepackage{multicol} 						%Pour le texte sur plusieurs colonnes
\usepackage[svgnames]{xcolor} 				%Pour utiliser des couleurs (marche pas avec libertine)
\usepackage{graphicx}						%Pour inclure des images
%\usepackage{listings}						%Pour écrire du code
\usepackage{lipsum}							%ajoute \lipsum
\usepackage{framed}							%Pour les encadrements
\usepackage{caption}						%Rajoute la fonction \caption*
\usepackage{tikz}							%Pour dessiner !
	\usetikzlibrary{patterns}				%Pour les hachures
\usepackage{marginnote}						%Pour des notes dans... ben la marge, quoi.
	\renewcommand*{\raggedleftmarginnote}{\centering}
	\renewcommand*{\raggedrightmarginnote}{\centering}
%\usepackage{etoolbox}


\usepackage{amsmath} 						%Trois paquets qui chargent les fontes mathématiques
\usepackage{amssymb}						%|de l'American Mathematical Society   
\usepackage{amsfonts}						%|
\usepackage{stmaryrd}						%Plus de symboles mathématiques ?	
%\usepackage{latexsym}						%D'autres symboles mathématiques.	
\usepackage{amsthm}							%Pour créer des théorèmes
\usepackage{bm}								%Bold Math
\usepackage{enumitem}						%Pour modifier numérotation de enumerate
\usepackage{array}							%Permet de faire des array en displaystyle
\usepackage{mathrsfs}						%Pour définir \mathscr{ }
\usepackage{nicefrac}						%ajoute \nicefrac{nom}{denom}
\usepackage{numprint}						%ajoute \numprint{nombre}
\usepackage{fourier-orns}					%Pour les tought break
\usepackage{extarrows}						%Pour avoir plus de flèches ! :D


%\usepackage{ifthen} 	  					%B-A BA de la programmation
%\usepackage{ifmtarg}						%Permet : if argument empty/if not empty
%\usepackage{multido}						%Inclu les boucles

\usepackage{vertthm}						%Ajoute \newVertTheorem et \newVertTheorem*
%	#1	Symbole de fin
%	#2	Nom de la macro
%	#3	Titre
%	#4	Couleur
%	#5	Style
%	#6 	Compteur, optionnel. Défaut : crée son propre compteur
%\newVertTheorem[Symbole de fin]{macro}{Titre}{Couleur}{style}{compteur}		OU
%\newVertTheorem{macro}{Titre}{Couleur}{style}{compteur}						OU
%\newVertTheorem[Symbole de fin]{macro}{Titre}{Couleur}{style}{}			 	OU
%\newVertTheorem{macro}{Titre}{Couleur}{style}{}								OU
%\newVertTheorem*[Symbole de fin]{macro}{Titre}{Couleur}{style}{}				OU
%\newVertTheorem*{macro}{Titre}{Couleur}{style}{}
%===========================================GENERAL=================================================
%Arguments-à-modifier-------------------------------------------------------------------------------
\title{Distribution des~$p$-sous-groupes de Sylow de la Jacobienne d'un graphe aléatoire}	%Titre

\def\authorOneFisrtName{Erkan}								%Prénom(s) de l'auteur 1
\def\authorOneFamilyName{Narmanli}							%Nom de famille de l'auteur 1
\def\authorTwoFisrtName{Ludovic}								%Prénom(s) de l'auteur 2
\def\authorTwoFamilyName{Stephan}								%Nom de famille de l'auteur 2
\def\supervisorFirstName{Omid}							%Prénom(s) du superviseur
\def\supervisorFamilyName{Amini}								%Nom de famille du superviseur
\def\instituteName{École normale supérieure}				%Nom de l'Université
\def\dateName{\today}										%Date

%Arguments-automatisés------------------------------------------------------------------------------ 
\author{\authorOneFisrtName{}~\authorOneFamilyName{} \& \authorTwoFisrtName{}~\authorTwoFamilyName{}}
\date{\today}								%Date du jour si non spécifié
%===========================================NUMEROTATION============================================
%		*Rappels----------------------------------------------------------------------------*
%		|\arabic	: nombres arabes					\fnsymbols	: syboles				|
%		|\roman 	: nombres romains minuscules		\alph 		: lettres minuscules	|
%		|\Roman 	: nombres romains majuscules		\Alph 		: lettres majuscules	|
%		*-----------------------------------------------------------------------------------*

\renewcommand{\thefootnote}{\alph{footnote}} 				%Notes de bas de page
\renewcommand{\theenumi}{\alph{enumi}}						%enviro. enumerate
%\setcounter{secnumdepth}{5}								%*A retrouver*
\renewcommand{\thesection}{\Alph{section}}					%Section
\renewcommand{\thesubsection}{\thesection.\arabic{subsection}}	%Sous-section
%\renewcommand{\thesubsubsection}{\alph{subsubsection}.}	%Sous-sous-section
%\renewcommand{\theparagraph}{}								%Paragraphe
%\renewcommand{\thesubparagraph}{(\roman{subparagraph})}	%Sous-paragraphe

%===========================================COULEURS================================================
\definecolor{bleu}{RGB}{34,66,124}
%\definecolor{rouge}{RGB}{169,17,1}
%\definecolor{vert}{RGB}{34,120,15}
\definecolor{grisDemo}{RGB}{112,111,108}
\definecolor{DemoBarre}{RGB}{225,223,218}
%\definecolor{noir}{RGB}{0,0,0}
%\definecolor{blanc}{RGB}{255,255,255}
%\definecolor{grisf}{RGB}{60,60,60}

%===========================================COMMANDES===============================================
%ThoughtBreak---------------------------------------------------------------------------------------
\newsavebox{\tbglyph}
\savebox{\tbglyph}[1.25in]{%
  %\hbox{\includegraphics[width=1.25in]{./images/tb.pdf}}
  \hbox{\LARGE\aldine}
}
\newcommand{\Tbreak}{% Thoughtbreaks are implemented as graphics
\vfil
  \begin{center}
    \usebox{\tbglyph}
  \end{center}
\vfil\vfil
}
%Opérateurs-mathématiques---------------------------------------------------------------------------
\DeclareMathOperator{\Res}{Rés}
\DeclareMathOperator{\B}{B}
\DeclareMathOperator{\Li}{Li}
\DeclareMathOperator{\Hom}{Hom}
%Sans-arguments-------------------------------------------------------------------------------------
\newcommand{\phiCF}{\phi_{C,F}}
\newcommand{\phiFC}{\phi_{F,C}}
\newcommand{\Vsig}{V_\sigma}
\newcommand{\Sk}{\mathbb{S}_k}								%Slab de taille k
\newcommand{\priv}{\setminus}								%Soustraction ensembliste
\renewcommand{\leq}{\leqslant}								%Parce que \leqslant est plus joli
\renewcommand{\geq}{\geqslant}								%Parce que \geqslant est plus joli
\renewcommand{\epsilon}{\varepsilon}						%Parce que \varepsilon est plus joli
\renewcommand{\Re}{\mathfrak{Re}}  							%Partie réelle
\renewcommand{\Im}{\mathfrak{Im}}  							%Partie imaginaire
\newcommand{\Pre}{\mathscr{P}}								%Ensemble des nombres premiers
%--
\makeatletter														
\newsavebox{\@brx}									%On définit les fonctions
\newcommand{\llangle}[1][]{\savebox{\@brx}{\(\m@th{#1\langle}\)}	%\rrangle et \llangle
  \mathopen{\copy\@brx\mkern2mu\kern-0.9\wd\@brx\usebox{\@brx}}}	%
\newcommand{\rrangle}[1][]{\savebox{\@brx}{\(\m@th{#1\rangle}\)}	%
  \mathclose{\copy\@brx\mkern2mu\kern-0.9\wd\@brx\usebox{\@brx}}}	%
\makeatother
%--
\renewcommand\cleardoublepage{\clearpage\ifodd\value{page}\else\null\clearpage\fi}
	%Pour recréer un \cleardoublepage qui marche aussi en oneside
%Avec-arguments-------------------------------------------------------------------------------------
\makeatletter
\newcommand{\restr}[1]{{}_{|#1}}
\newcommand{\Cmin}[1][\omega]{\gamma_{\mathrm{min}}(#1)}%Chemin minimal de \gamma
\newcommand{\ZnB}[1][n]{\overline{Z}_{#1}}		%L'ensemble Z_n barre
\newcommand{\SnB}[1][3n]{\overline{S}_{#1}}		%S_{3n} /!\ arg. par défaut : 3n
\newcommand{\BnB}[1][n]{\overline{B}_{#1}}		%B_n
\newcommand{\BRz}[1][2]{\overline{B_{#1}}(z)}	%B_R(z)
\newcommand{\Pro}{\bm{\mathrm{P}}}				%Le 'P' de la probabilité
\newcommand{\PpSP}[1][p]{\Pro_{#1}}				%Proba indexée par un paramètre
\newcommand{\Pp}[2][p]{\Pro_{#1}\left[#2\right]}		%Proba ind. par p avec des crochets
\newcommand{\relie}{\@ifstar{\relie@STAR}{\relie@NOSTAR}}	%event : relier #2 à #3 en passant par #1
\newcommand{\relie@NOSTAR}[3][]{#2\xleftrightarrow{#1}#3}
\newcommand{\relie@STAR}[3][]{#2\xleftrightarrow{!#1!}#3}
\newcommand{\siecle}[1]{{\sc\romannumeral #1}\ifnum#1=1\relax{}\ier\else\ieme\fi}
	%\siècle{nb}
	%NB : gère automatiquement le cas Ier siècle ou le reste (ex : IIIe)
\newcommand{\norme}[1]{\left\Vert #1\right\Vert}			%Norme (deux doubles barres)
\newcommand{\abs}[1]{\left\vert #1\right\vert}				%Valeur absolue (deux barres simples)
\newcommand{\entier}[1]{\lfloor #1\rfloor}					%Partie entière inférieure
\newcommand{\entierSup}[1]{\lceil #1\rceil}					%Partie entière supérieure
\newcommand{\grandO}[1]{O\left(#1\right)}					%Grand O
\newcommand{\petitO}[1]{o\left(#1\right)}					%petit o
\newcommand\suite{\@ifstar{\suite@STAR}{\suite@NOSTAR}}				
\newcommand{\suite@NOSTAR}[2][n]{\left(#2_{#1}\right)_{#1\in\mathbb{N}}}
\newcommand{\suite@STAR}[2][n]{\left(#2_{#1}\right)_{#1\in\mathbb{N}^*}}				
	%\suite[i]{a} 	pour obtenir (a_i)_{i\in N}				OU
	%\suite{a}		pour obtenir (a_n)_{n\in N}
	%|NB : l'argument optionnel sert juste a modifier l'indice, n par défaut.
\newcommand{\demsection}[1]{\noindent\textit{\textcolor{grisDemo}{#1.}}}
	%\demsection{titre}
	%|NB : se contente de mettre en page les titres des parties d'un démonstration
\newcommand{\personne}[5][]{\footnote{#2 #3, #1 (#4--#5).}}
	%\personne[notes]{prenom}{nom}{date_de_naissance}{date_de_mort}
	%|NB : la personne doit être morte. && Le met automatiquement en bas de page.
\newcommand{\mesnotes}[1]{\textcolor{bleu}{(#1)}}
%\newcommand{\mesnotes}[1]{}
	%\mesnote{texte}
	%|NB : colorie le texte en bleu et l'encadre de parenthèses
	%|NB : ne pas oublier de les effacer à la fin !
\newcommand{\draftnote}[1]{\textcolor{red}{[(#1)]}}
	%\draftnote{texte}
	%|NB : colorie le texte en un rouge qui fait bien mal aux yeux et l'encadre de parenthèses
	%|NB : ne pas oublier de les effacer à la fin !
\makeatother
%Nouvelles-colonnes---------------------------------------------------------------------------------
%\newcolumntype{L}{>{\displaystyle}l}						%Colonne alignée left en displaystyle
%\newcolumntype{C}{>{\displaystyle}c}						%Colonne alignée center en displaystyle
%\newcolumntype{R}{>{\displaystyle}r}						%Colonne alignée right en displaystyle
%===========================================ENVIRONNEMENTS==========================================
%\makeatletter
%\patchcmd{\@mn@margintest}{\@tempswafalse}{\@tempswatrue}{}{}
%\patchcmd{\@mn@margintest}{\@tempswafalse}{\@tempswatrue}{}{}
%\reversemarginpar 
%\makeatother
%Theorèmes------------------------------------------------------------------------------------------
\theoremstyle{definition}
	\newtheorem{defi}{Définition} [section]
\theoremstyle{plain}
	\newtheorem{thm}[defi]{Théorème}
	\newtheorem{prop}[defi]{Proposition}
	\newtheorem{lem}[defi]{Lemme}
	\newtheorem{coro}[defi]{Corollaire}
	\newtheorem*{conj}{Conjecture}
\theoremstyle{remark}
	\newtheorem*{rem}{Remarque}
	\newtheorem*{voca}{Vocabulaire}
	\newtheorem{ex}{Exemple}
	%\newVertTheorem*[$\blacksquare$]{dem}{Démonstration}{DemoBarre}{\itshape}{defi}
	

%Autres---------------------------------------------------------------------------------------------
\renewcommand{\qedsymbol}{\rotatebox{45}{\tiny\ensuremath{\blacksquare}}}
\newenvironment{dem}[1][]
	{
		\ifcat$\detokenize{#1}$
			\begin{proof}
		\else
			\begin{proof}[\proofname\ {\normalfont (#1)}]
		\fi
		%\advance\leftskip by 0.5cm
	}{
		\end{proof}
	}
\newenvironment{quartCarre}[1][0.3]
	{%
		\begin{tikzpicture}[scale = #1]
			%Clip
			\clip 					(-1.9,-1.9)	rectangle	(18,18)	;
			%Fond
			\fill	[color=gray!8]	(-15,-15)	rectangle	(15,15)	;	%B_n
			%Rectangles
			\draw 	[very thick]	(-15,-15)	rectangle 	(15,15) ;	%B_3n
			\draw 					(-2,-2) 	rectangle 	(2,2) 	;	%S_3n
			\draw 	[thick]			(5, 3)		rectangle	(15,13)	;	%B_n'
			\draw 					(9, 7)		rectangle	(11,9)	;	%S_n'
			%Points
			\path	[below right]	(-1.9, 15)	node				{$B_{3_n}$}	;
			\path	[below right]	(-2, 2)		node	[scale=0.85]{$S_{3n}$}	;
			\path	[below right]	(5, 13)		node	[scale=0.85]{$B_n'$}	;
			\path	[below right]	(9, 9)		node	[scale=0.7]	{$S_n'$}	;
	}{%

		\end{tikzpicture}
	}%
%Bibibliographie------------------------------------------------------------------------------------
\makeatletter
\renewenvironment{thebibliography}[1]
     {\vspace*{1cm}
     {\LARGE\bf Bibliographie}%
      
     \vspace*{0.5cm}
      \list{\@biblabel{\@arabic\c@enumiv}}%
           {\settowidth\labelwidth{\@biblabel{#1}}%
            \leftmargin\labelwidth
            \advance\leftmargin\labelsep
            \@openbib@code
            \usecounter{enumiv}%
            \let\p@enumiv\@empty
            \renewcommand\theenumiv{\@arabic\c@enumiv}}%
      \sloppy
      \clubpenalty4000
      \@clubpenalty \clubpenalty
      \widowpenalty4000%
      \sfcode`\.\@m}
     {\def\@noitemerr
       {\@latex@warning{Empty `thebibliography' environment}}%
      \endlist}
\makeatother
