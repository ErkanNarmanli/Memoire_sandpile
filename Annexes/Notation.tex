\vspace*{0.3cm}
	Il s'agit donc d'étudier les Jacobiennes des graphes tirés selon la distribution d'Erd\H{o}s-Rényi. Soient~$n$ un entier non nul et~$q$\marginnote{$n,q$} un réel strictement compris entre~$0$ et~$1$ ; un graphe~$G$ est tiré selon la distribution d'Erd\H{o}s-Rényi lorsque toute paire de sommets de~$G$ forme une arête avec probabilité~$q$, indépendamment des autres arrêtes dans le graphe ; on notera alors~$G\in G(n,q)$\marginnote{$G(n,q)$} pour dire que~$G$ a été tiré selon cette distribution.
	
	Si~$G$ est un graphe fini connexe, on notera~$L_G$\marginnote{$L_G$} le laplacien de~$G$. Dès lors, si on note~$\Div^0(G)$\marginnote{$\Div^0(G)$} l'ensemble des diviseurs de~$G$ de degré nulle, on a clairement l'inclusion~$\Im(L_\Gamma) \subset \Div^0(G)$. On définit alors la \emph{Jacobienne} de~$G$ comme étant le groupe~$\Jac(G)$ donné par :
	\[
		\Jac(G) = \quotient{\Div^0(G)}{\Princ(G)}.\text{\marginnote{$\Jac(G)$}}
	\]
	où,~$\Princ(G)$\marginnote{$\Princ(G)$} désigne l'ensemble des diviseurs des fonction méromorphes sur~$G$. On notera~$\mathcal{M}(G)$\marginnote{$\mathcal{M}(G)$} l'ensemble des fonctions méromorphes sur~$G$.
	
	 Si~$G$ n'est pas connexe, on définira la jacobienne de~$G$ comme étant la somme directe des jacobiennes de ses composantes connexes. 