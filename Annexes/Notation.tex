\vspace*{0.3cm}
	Il s'agit donc d'étudier les Jacobiennes des graphes tirés selon la distribution d'Erd\H{o}s-Rényi. Soient~$n$ un entier non nul et~$q$\marginnote{$n,q$} un réel strictement compris entre~$0$ et~$1$,un graphe~$\Gamma$ est tiré selon la distribution d'Erd\H{o}s-Rényi lorsque pour toute paire de sommets de~$G$ forme une arrête avec probabilité~$q$, indépendamment des autres arrêtes dans le graphe ; on notera alors~$G\in G(n,q)$\marginnote{$G(n,q)$} pour dire que~$G$ a été tiré selon cette distribution.
	
	Si~$G$ est un graphe, on rappelle qu'on peut lui associer son \emph{Laplacien}~$L_G$ \marginnote{$L_G$}, la matrice de taille~$n\times n$ à coefficients dans~$\mathbb{Z}$ et donnée par :
	\[
		\forall i,j\in[n], L_G(i,j) = 
			\begin{cases}
				1 &\text{lorsque }\{i,j\}\text{ est une arrête de }G\\
				-1 &\text{si }i=j\\
				0 &\text{sinon}
			\end{cases}
	\]
	Si on note~$\Div^0(G)$\marginnote{$\Div^0(G)$} l'ensemble des diviseurs de~$G$ de degré nulle, alors on a clairement que~$\Im(L_\Gamma) \subset \Div^0(G)$. On définit alors la \emph{Jacobienne} de~$G$ comme étant le groupe~$\Jac(G)$ donné par :
	\[
		\Jac(G) = \quotient{\Div^0(G)}{\Princ(G)}.\text{\marginnote{$\Jac(G)$}}
	\]
	où,~$\Princ(G)$ désigne l'ensemble des diviseurs des fonction méromorphes sur~$G$.