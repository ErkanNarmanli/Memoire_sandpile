\vspace*{0.3cm}
	Il s'agit donc d'étudier les Jacobiennes des graphes tirés selon la distribution d'Erd\H{o}s-Rényi. Soient~$n$ un entier non nul et~$q$\marginnote{$n,q$} un réel strictement compris entre~$0$ et~$1$,un graphe~$\Gamma$ est tiré selon la distribution d'Erd\H{o}s-Rényi lorsque pour toute paire de sommets de~$\Gamma$ forme une arrête avec probabilité~$q$, indépendamment des autres arrêtes dans le graphe ; on notera alors~$\Gamma\in G(n,q)$\marginnote{$G(n,q)$} pour dire que~$\Gamma$ a été tiré selon cette distribution.
	
	Dans la suite de cet exposé, et pour alléger les notations, on écrira~$[n]$\marginnote{$[n]$} pour désigner l'ensemble~$\{1,2,...,n\}$ des~$n$ premiers entiers naturels non nuls.
	
	Si~$\Gamma$ est un graphe, on rappelle qu'on peut lui associer son \emph{Laplacien}~$L_\Gamma$ \marginnote{$L_\Gamma$}, la matrice de taille~$n\times n$ à coefficients dans~$\mathbb{Z}$ et donnée par :
	\[
		\forall i,j\in[n], L_\Gamma(i,j) = 
			\begin{cases}
				1 &\text{lorsque }\{i,j\}\text{ est une arrête de }\Gamma \\
				-1 &\text{si }i=j\\
				0 &\text{sinon}
			\end{cases}
	\]
	Si on note~$\Zzero$\marginnote{$\Zzero$} l'ensemble des vecteurs de~$\mathbb{Z}^n$ dont la somme des coordonnées est nulle, alors on a clairement que~$\Im(L_\Gamma) \subset \Zzero$. On définit alors la \emph{Jacobienne} de~$\Gamma$ comme étant le groupe~$S_\Gamma$ donné par :
	\[
		S_\Gamma = \quotient{\Zzero}{\Im(L_\Gamma)}.\text{\marginnote{$S_\Gamma$}}
	\]